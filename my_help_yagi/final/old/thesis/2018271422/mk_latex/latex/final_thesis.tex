
% Default to the notebook output style

    


% Inherit from the specified cell style.




    
\documentclass[11pt,dvipdfmx]{jsarticle}

    
    
    \usepackage[T1]{fontenc}
    % Nicer default font (+ math font) than Computer Modern for most use cases
    \usepackage{mathpazo}

    % Basic figure setup, for now with no caption control since it's done
    % automatically by Pandoc (which extracts ![](path) syntax from Markdown).
   \usepackage{wrapfig}
    \usepackage{graphicx}
    % We will generate all images so they have a width \maxwidth. This means
    % that they will get their normal width if they fit onto the page, but
    % are scaled down if they would overflow the margins.
    \makeatletter
    \def\maxwidth{\ifdim\Gin@nat@width>\linewidth\linewidth
    \else\Gin@nat@width\fi}
    \makeatother
    \let\Oldincludegraphics\includegraphics
    % Set max figure width to be 80% of text width, for now hardcoded.
%    \renewcommand{\includegraphics}[1]{\Oldincludegraphics[width=.8\maxwidth]{#1}}
    % Ensure that by default, figures have no caption (until we provide a
    % proper Figure object with a Caption API and a way to capture that
    % in the conversion process - todo).
    \usepackage{caption}
%    \DeclareCaptionLabelFormat{nolabel}{}
%    \captionsetup{labelformat=nolabel}

    \usepackage{adjustbox} % Used to constrain images to a maximum size 
    \usepackage{xcolor} % Allow colors to be defined
    \usepackage{enumerate} % Needed for markdown enumerations to work
    \usepackage{geometry} % Used to adjust the document margins
    \usepackage{amsmath} % Equations
    \usepackage{amssymb} % Equations
    \usepackage{textcomp} % defines textquotesingle
    % Hack from http://tex.stackexchange.com/a/47451/13684:
    \AtBeginDocument{%
        \def\PYZsq{\textquotesingle}% Upright quotes in Pygmentized code
    }
    \usepackage{upquote} % Upright quotes for verbatim code
    \usepackage{eurosym} % defines \euro
    \usepackage[mathletters]{ucs} % Extended unicode (utf-8) support
    \usepackage[utf8x]{inputenc} % Allow utf-8 characters in the tex document
    \usepackage{fancyvrb} % verbatim replacement that allows latex
    \usepackage{grffile} % extends the file name processing of package graphics 
                         % to support a larger range 
    % The hyperref package gives us a pdf with properly built
    % internal navigation ('pdf bookmarks' for the table of contents,
    % internal cross-reference links, web links for URLs, etc.)
    \usepackage{hyperref}
    \usepackage{longtable} % longtable support required by pandoc >1.10
    \usepackage{booktabs}  % table support for pandoc > 1.12.2
    \usepackage[inline]{enumitem} % IRkernel/repr support (it uses the enumerate* environment)
    \usepackage[normalem]{ulem} % ulem is needed to support strikethroughs (\sout)
                                % normalem makes italics be italics, not underlines
    

    
    
    % Colors for the hyperref package
    \definecolor{urlcolor}{rgb}{0,.145,.698}
    \definecolor{linkcolor}{rgb}{.71,0.21,0.01}
    \definecolor{citecolor}{rgb}{.12,.54,.11}

    % ANSI colors
    \definecolor{ansi-black}{HTML}{3E424D}
    \definecolor{ansi-black-intense}{HTML}{282C36}
    \definecolor{ansi-red}{HTML}{E75C58}
    \definecolor{ansi-red-intense}{HTML}{B22B31}
    \definecolor{ansi-green}{HTML}{00A250}
    \definecolor{ansi-green-intense}{HTML}{007427}
    \definecolor{ansi-yellow}{HTML}{DDB62B}
    \definecolor{ansi-yellow-intense}{HTML}{B27D12}
    \definecolor{ansi-blue}{HTML}{208FFB}
    \definecolor{ansi-blue-intense}{HTML}{0065CA}
    \definecolor{ansi-magenta}{HTML}{D160C4}
    \definecolor{ansi-magenta-intense}{HTML}{A03196}
    \definecolor{ansi-cyan}{HTML}{60C6C8}
    \definecolor{ansi-cyan-intense}{HTML}{258F8F}
    \definecolor{ansi-white}{HTML}{C5C1B4}
    \definecolor{ansi-white-intense}{HTML}{A1A6B2}

    % commands and environments needed by pandoc snippets
    % extracted from the output of `pandoc -s`
    \providecommand{\tightlist}{%
      \setlength{\itemsep}{0pt}\setlength{\parskip}{0pt}}
    \DefineVerbatimEnvironment{Highlighting}{Verbatim}{commandchars=\\\{\}}
    % Add ',fontsize=\small' for more characters per line
    \newenvironment{Shaded}{}{}
    \newcommand{\KeywordTok}[1]{\textcolor[rgb]{0.00,0.44,0.13}{\textbf{{#1}}}}
    \newcommand{\DataTypeTok}[1]{\textcolor[rgb]{0.56,0.13,0.00}{{#1}}}
    \newcommand{\DecValTok}[1]{\textcolor[rgb]{0.25,0.63,0.44}{{#1}}}
    \newcommand{\BaseNTok}[1]{\textcolor[rgb]{0.25,0.63,0.44}{{#1}}}
    \newcommand{\FloatTok}[1]{\textcolor[rgb]{0.25,0.63,0.44}{{#1}}}
    \newcommand{\CharTok}[1]{\textcolor[rgb]{0.25,0.44,0.63}{{#1}}}
    \newcommand{\StringTok}[1]{\textcolor[rgb]{0.25,0.44,0.63}{{#1}}}
    \newcommand{\CommentTok}[1]{\textcolor[rgb]{0.38,0.63,0.69}{\textit{{#1}}}}
    \newcommand{\OtherTok}[1]{\textcolor[rgb]{0.00,0.44,0.13}{{#1}}}
    \newcommand{\AlertTok}[1]{\textcolor[rgb]{1.00,0.00,0.00}{\textbf{{#1}}}}
    \newcommand{\FunctionTok}[1]{\textcolor[rgb]{0.02,0.16,0.49}{{#1}}}
    \newcommand{\RegionMarkerTok}[1]{{#1}}
    \newcommand{\ErrorTok}[1]{\textcolor[rgb]{1.00,0.00,0.00}{\textbf{{#1}}}}
    \newcommand{\NormalTok}[1]{{#1}}
    
    % Additional commands for more recent versions of Pandoc
    \newcommand{\ConstantTok}[1]{\textcolor[rgb]{0.53,0.00,0.00}{{#1}}}
    \newcommand{\SpecialCharTok}[1]{\textcolor[rgb]{0.25,0.44,0.63}{{#1}}}
    \newcommand{\VerbatimStringTok}[1]{\textcolor[rgb]{0.25,0.44,0.63}{{#1}}}
    \newcommand{\SpecialStringTok}[1]{\textcolor[rgb]{0.73,0.40,0.53}{{#1}}}
    \newcommand{\ImportTok}[1]{{#1}}
    \newcommand{\DocumentationTok}[1]{\textcolor[rgb]{0.73,0.13,0.13}{\textit{{#1}}}}
    \newcommand{\AnnotationTok}[1]{\textcolor[rgb]{0.38,0.63,0.69}{\textbf{\textit{{#1}}}}}
    \newcommand{\CommentVarTok}[1]{\textcolor[rgb]{0.38,0.63,0.69}{\textbf{\textit{{#1}}}}}
    \newcommand{\VariableTok}[1]{\textcolor[rgb]{0.10,0.09,0.49}{{#1}}}
    \newcommand{\ControlFlowTok}[1]{\textcolor[rgb]{0.00,0.44,0.13}{\textbf{{#1}}}}
    \newcommand{\OperatorTok}[1]{\textcolor[rgb]{0.40,0.40,0.40}{{#1}}}
    \newcommand{\BuiltInTok}[1]{{#1}}
    \newcommand{\ExtensionTok}[1]{{#1}}
    \newcommand{\PreprocessorTok}[1]{\textcolor[rgb]{0.74,0.48,0.00}{{#1}}}
    \newcommand{\AttributeTok}[1]{\textcolor[rgb]{0.49,0.56,0.16}{{#1}}}
    \newcommand{\InformationTok}[1]{\textcolor[rgb]{0.38,0.63,0.69}{\textbf{\textit{{#1}}}}}
    \newcommand{\WarningTok}[1]{\textcolor[rgb]{0.38,0.63,0.69}{\textbf{\textit{{#1}}}}}
    
    
    % Define a nice break command that doesn't care if a line doesn't already
    % exist.
    \def\br{\hspace*{\fill} \\* }
    % Math Jax compatability definitions
    \def\gt{>}
    \def\lt{<}
    % Document parameters
    \title{final\_thesis}
    
    
    

    % Pygments definitions
    
\makeatletter
\def\PY@reset{\let\PY@it=\relax \let\PY@bf=\relax%
    \let\PY@ul=\relax \let\PY@tc=\relax%
    \let\PY@bc=\relax \let\PY@ff=\relax}
\def\PY@tok#1{\csname PY@tok@#1\endcsname}
\def\PY@toks#1+{\ifx\relax#1\empty\else%
    \PY@tok{#1}\expandafter\PY@toks\fi}
\def\PY@do#1{\PY@bc{\PY@tc{\PY@ul{%
    \PY@it{\PY@bf{\PY@ff{#1}}}}}}}
\def\PY#1#2{\PY@reset\PY@toks#1+\relax+\PY@do{#2}}

\expandafter\def\csname PY@tok@w\endcsname{\def\PY@tc##1{\textcolor[rgb]{0.73,0.73,0.73}{##1}}}
\expandafter\def\csname PY@tok@c\endcsname{\let\PY@it=\textit\def\PY@tc##1{\textcolor[rgb]{0.25,0.50,0.50}{##1}}}
\expandafter\def\csname PY@tok@cp\endcsname{\def\PY@tc##1{\textcolor[rgb]{0.74,0.48,0.00}{##1}}}
\expandafter\def\csname PY@tok@k\endcsname{\let\PY@bf=\textbf\def\PY@tc##1{\textcolor[rgb]{0.00,0.50,0.00}{##1}}}
\expandafter\def\csname PY@tok@kp\endcsname{\def\PY@tc##1{\textcolor[rgb]{0.00,0.50,0.00}{##1}}}
\expandafter\def\csname PY@tok@kt\endcsname{\def\PY@tc##1{\textcolor[rgb]{0.69,0.00,0.25}{##1}}}
\expandafter\def\csname PY@tok@o\endcsname{\def\PY@tc##1{\textcolor[rgb]{0.40,0.40,0.40}{##1}}}
\expandafter\def\csname PY@tok@ow\endcsname{\let\PY@bf=\textbf\def\PY@tc##1{\textcolor[rgb]{0.67,0.13,1.00}{##1}}}
\expandafter\def\csname PY@tok@nb\endcsname{\def\PY@tc##1{\textcolor[rgb]{0.00,0.50,0.00}{##1}}}
\expandafter\def\csname PY@tok@nf\endcsname{\def\PY@tc##1{\textcolor[rgb]{0.00,0.00,1.00}{##1}}}
\expandafter\def\csname PY@tok@nc\endcsname{\let\PY@bf=\textbf\def\PY@tc##1{\textcolor[rgb]{0.00,0.00,1.00}{##1}}}
\expandafter\def\csname PY@tok@nn\endcsname{\let\PY@bf=\textbf\def\PY@tc##1{\textcolor[rgb]{0.00,0.00,1.00}{##1}}}
\expandafter\def\csname PY@tok@ne\endcsname{\let\PY@bf=\textbf\def\PY@tc##1{\textcolor[rgb]{0.82,0.25,0.23}{##1}}}
\expandafter\def\csname PY@tok@nv\endcsname{\def\PY@tc##1{\textcolor[rgb]{0.10,0.09,0.49}{##1}}}
\expandafter\def\csname PY@tok@no\endcsname{\def\PY@tc##1{\textcolor[rgb]{0.53,0.00,0.00}{##1}}}
\expandafter\def\csname PY@tok@nl\endcsname{\def\PY@tc##1{\textcolor[rgb]{0.63,0.63,0.00}{##1}}}
\expandafter\def\csname PY@tok@ni\endcsname{\let\PY@bf=\textbf\def\PY@tc##1{\textcolor[rgb]{0.60,0.60,0.60}{##1}}}
\expandafter\def\csname PY@tok@na\endcsname{\def\PY@tc##1{\textcolor[rgb]{0.49,0.56,0.16}{##1}}}
\expandafter\def\csname PY@tok@nt\endcsname{\let\PY@bf=\textbf\def\PY@tc##1{\textcolor[rgb]{0.00,0.50,0.00}{##1}}}
\expandafter\def\csname PY@tok@nd\endcsname{\def\PY@tc##1{\textcolor[rgb]{0.67,0.13,1.00}{##1}}}
\expandafter\def\csname PY@tok@s\endcsname{\def\PY@tc##1{\textcolor[rgb]{0.73,0.13,0.13}{##1}}}
\expandafter\def\csname PY@tok@sd\endcsname{\let\PY@it=\textit\def\PY@tc##1{\textcolor[rgb]{0.73,0.13,0.13}{##1}}}
\expandafter\def\csname PY@tok@si\endcsname{\let\PY@bf=\textbf\def\PY@tc##1{\textcolor[rgb]{0.73,0.40,0.53}{##1}}}
\expandafter\def\csname PY@tok@se\endcsname{\let\PY@bf=\textbf\def\PY@tc##1{\textcolor[rgb]{0.73,0.40,0.13}{##1}}}
\expandafter\def\csname PY@tok@sr\endcsname{\def\PY@tc##1{\textcolor[rgb]{0.73,0.40,0.53}{##1}}}
\expandafter\def\csname PY@tok@ss\endcsname{\def\PY@tc##1{\textcolor[rgb]{0.10,0.09,0.49}{##1}}}
\expandafter\def\csname PY@tok@sx\endcsname{\def\PY@tc##1{\textcolor[rgb]{0.00,0.50,0.00}{##1}}}
\expandafter\def\csname PY@tok@m\endcsname{\def\PY@tc##1{\textcolor[rgb]{0.40,0.40,0.40}{##1}}}
\expandafter\def\csname PY@tok@gh\endcsname{\let\PY@bf=\textbf\def\PY@tc##1{\textcolor[rgb]{0.00,0.00,0.50}{##1}}}
\expandafter\def\csname PY@tok@gu\endcsname{\let\PY@bf=\textbf\def\PY@tc##1{\textcolor[rgb]{0.50,0.00,0.50}{##1}}}
\expandafter\def\csname PY@tok@gd\endcsname{\def\PY@tc##1{\textcolor[rgb]{0.63,0.00,0.00}{##1}}}
\expandafter\def\csname PY@tok@gi\endcsname{\def\PY@tc##1{\textcolor[rgb]{0.00,0.63,0.00}{##1}}}
\expandafter\def\csname PY@tok@gr\endcsname{\def\PY@tc##1{\textcolor[rgb]{1.00,0.00,0.00}{##1}}}
\expandafter\def\csname PY@tok@ge\endcsname{\let\PY@it=\textit}
\expandafter\def\csname PY@tok@gs\endcsname{\let\PY@bf=\textbf}
\expandafter\def\csname PY@tok@gp\endcsname{\let\PY@bf=\textbf\def\PY@tc##1{\textcolor[rgb]{0.00,0.00,0.50}{##1}}}
\expandafter\def\csname PY@tok@go\endcsname{\def\PY@tc##1{\textcolor[rgb]{0.53,0.53,0.53}{##1}}}
\expandafter\def\csname PY@tok@gt\endcsname{\def\PY@tc##1{\textcolor[rgb]{0.00,0.27,0.87}{##1}}}
\expandafter\def\csname PY@tok@err\endcsname{\def\PY@bc##1{\setlength{\fboxsep}{0pt}\fcolorbox[rgb]{1.00,0.00,0.00}{1,1,1}{\strut ##1}}}
\expandafter\def\csname PY@tok@kc\endcsname{\let\PY@bf=\textbf\def\PY@tc##1{\textcolor[rgb]{0.00,0.50,0.00}{##1}}}
\expandafter\def\csname PY@tok@kd\endcsname{\let\PY@bf=\textbf\def\PY@tc##1{\textcolor[rgb]{0.00,0.50,0.00}{##1}}}
\expandafter\def\csname PY@tok@kn\endcsname{\let\PY@bf=\textbf\def\PY@tc##1{\textcolor[rgb]{0.00,0.50,0.00}{##1}}}
\expandafter\def\csname PY@tok@kr\endcsname{\let\PY@bf=\textbf\def\PY@tc##1{\textcolor[rgb]{0.00,0.50,0.00}{##1}}}
\expandafter\def\csname PY@tok@bp\endcsname{\def\PY@tc##1{\textcolor[rgb]{0.00,0.50,0.00}{##1}}}
\expandafter\def\csname PY@tok@fm\endcsname{\def\PY@tc##1{\textcolor[rgb]{0.00,0.00,1.00}{##1}}}
\expandafter\def\csname PY@tok@vc\endcsname{\def\PY@tc##1{\textcolor[rgb]{0.10,0.09,0.49}{##1}}}
\expandafter\def\csname PY@tok@vg\endcsname{\def\PY@tc##1{\textcolor[rgb]{0.10,0.09,0.49}{##1}}}
\expandafter\def\csname PY@tok@vi\endcsname{\def\PY@tc##1{\textcolor[rgb]{0.10,0.09,0.49}{##1}}}
\expandafter\def\csname PY@tok@vm\endcsname{\def\PY@tc##1{\textcolor[rgb]{0.10,0.09,0.49}{##1}}}
\expandafter\def\csname PY@tok@sa\endcsname{\def\PY@tc##1{\textcolor[rgb]{0.73,0.13,0.13}{##1}}}
\expandafter\def\csname PY@tok@sb\endcsname{\def\PY@tc##1{\textcolor[rgb]{0.73,0.13,0.13}{##1}}}
\expandafter\def\csname PY@tok@sc\endcsname{\def\PY@tc##1{\textcolor[rgb]{0.73,0.13,0.13}{##1}}}
\expandafter\def\csname PY@tok@dl\endcsname{\def\PY@tc##1{\textcolor[rgb]{0.73,0.13,0.13}{##1}}}
\expandafter\def\csname PY@tok@s2\endcsname{\def\PY@tc##1{\textcolor[rgb]{0.73,0.13,0.13}{##1}}}
\expandafter\def\csname PY@tok@sh\endcsname{\def\PY@tc##1{\textcolor[rgb]{0.73,0.13,0.13}{##1}}}
\expandafter\def\csname PY@tok@s1\endcsname{\def\PY@tc##1{\textcolor[rgb]{0.73,0.13,0.13}{##1}}}
\expandafter\def\csname PY@tok@mb\endcsname{\def\PY@tc##1{\textcolor[rgb]{0.40,0.40,0.40}{##1}}}
\expandafter\def\csname PY@tok@mf\endcsname{\def\PY@tc##1{\textcolor[rgb]{0.40,0.40,0.40}{##1}}}
\expandafter\def\csname PY@tok@mh\endcsname{\def\PY@tc##1{\textcolor[rgb]{0.40,0.40,0.40}{##1}}}
\expandafter\def\csname PY@tok@mi\endcsname{\def\PY@tc##1{\textcolor[rgb]{0.40,0.40,0.40}{##1}}}
\expandafter\def\csname PY@tok@il\endcsname{\def\PY@tc##1{\textcolor[rgb]{0.40,0.40,0.40}{##1}}}
\expandafter\def\csname PY@tok@mo\endcsname{\def\PY@tc##1{\textcolor[rgb]{0.40,0.40,0.40}{##1}}}
\expandafter\def\csname PY@tok@ch\endcsname{\let\PY@it=\textit\def\PY@tc##1{\textcolor[rgb]{0.25,0.50,0.50}{##1}}}
\expandafter\def\csname PY@tok@cm\endcsname{\let\PY@it=\textit\def\PY@tc##1{\textcolor[rgb]{0.25,0.50,0.50}{##1}}}
\expandafter\def\csname PY@tok@cpf\endcsname{\let\PY@it=\textit\def\PY@tc##1{\textcolor[rgb]{0.25,0.50,0.50}{##1}}}
\expandafter\def\csname PY@tok@c1\endcsname{\let\PY@it=\textit\def\PY@tc##1{\textcolor[rgb]{0.25,0.50,0.50}{##1}}}
\expandafter\def\csname PY@tok@cs\endcsname{\let\PY@it=\textit\def\PY@tc##1{\textcolor[rgb]{0.25,0.50,0.50}{##1}}}

\def\PYZbs{\char`\\}
\def\PYZus{\char`\_}
\def\PYZob{\char`\{}
\def\PYZcb{\char`\}}
\def\PYZca{\char`\^}
\def\PYZam{\char`\&}
\def\PYZlt{\char`\<}
\def\PYZgt{\char`\>}
\def\PYZsh{\char`\#}
\def\PYZpc{\char`\%}
\def\PYZdl{\char`\$}
\def\PYZhy{\char`\-}
\def\PYZsq{\char`\'}
\def\PYZdq{\char`\"}
\def\PYZti{\char`\~}
% for compatibility with earlier versions
\def\PYZat{@}
\def\PYZlb{[}
\def\PYZrb{]}
\makeatother


    % Exact colors from NB
    \definecolor{incolor}{rgb}{0.0, 0.0, 0.5}
    \definecolor{outcolor}{rgb}{0.545, 0.0, 0.0}



    
    % Prevent overflowing lines due to hard-to-break entities
    \sloppy 
    % Setup hyperref package
    \hypersetup{
      breaklinks=true,  % so long urls are correctly broken across lines
      colorlinks=true,
      urlcolor=urlcolor,
      linkcolor=linkcolor,
      citecolor=citecolor,
      }
    % Slightly bigger margins than the latex defaults
    
    \geometry{verbose,tmargin=1in,bmargin=1in,lmargin=1in,rmargin=1in}
    
    

    \begin{document}
    
    
    \maketitle
    
    

    
    Table of Contents{}

{{1~~}はじめに}

{{1.1~~}目的}

{{1.2~~}コマンドの重要性}

{{2~~}基本的事項}

{{2.1~~}my\_help}

{{2.1.1~~}目的}

{{2.1.2~~}コマンド}

{{2.2~~}optparse}

{{2.3~~}thor}

{{2.4~~}awesome}

{{3~~}研究内容}

{{3.1~~}thor化に伴う書き換え(具体的な作業)}

{{3.2~~}thorによるサブコマンド}

{{3.3~~}exeディレクトリの書き換え}

{{3.4~~}libディレクトリ}

{{3.5~~}その他オプション}

{{3.6~~}thorで実装できない部分と対処}

{{4~~}比較}

{{5~~}総括}

    Introduction Method Results and discussion

    \section{はじめに}\label{ux306fux3058ux3081ux306b}

\subsection{目的}\label{ux76eeux7684}

西谷研究室で使われているユーザメモソフト,my
helpの振る舞いを制御しているサブコマンドは,マイナスを付した省略記法が取られている.プログラミング初心者にとってこの省略記法は,覚えにくかったりわかりにくかったりするという問題があり,現在はフルワードを使った自然言語に近い記述法が多く用いられている.
そこで,本研究ではコマンドラインツール作成ライブラリを自然言語に近いサブコマンド体系を実装しやすいライブラリであるThorに変更する.rubyの標準ライブラリであるoptparseで作成されているmy
helpを,Thorによって書き直し,異なった2つライブラリで作成されたmy
helpの使用感を比較検討することを目的とする.

\subsection{コマンドの重要性}\label{ux30b3ux30deux30f3ux30c9ux306eux91cdux8981ux6027}

今日,複雑な機能を持つコマンドが増加している.そのようなコマンドは,サブコマンドを使用することで適切な動作を実行することが可能になる.例えばgitコマンドはサブコマンドなしでは意味をなさない.サブコマンドでどのような動作をするかが決まるので,サブコマンドを入力することで正常に動作するのである.

そもそも,サブコマンド(オプション)にはショートオプションとロングオプションの2種類がある.ショートオプションはハイフンの後に英字1字を付けた形式のもので,-aや-vなどといったものがショートオプションである.また,ショートオプションは2つ以上のオプションを1つにまとめて実行することもできる.例えば,-l,-a,-tの3つのオプションを1つにまとめて-latとして実行することが可能である.それに対してロングオプションは,ハイフン2つの後に英字2字以上を付けることができる形式である.例えば-\/-allや-\/-versionなどである.ロングオプションは英字を2字以上使用することができるので,どのようなオプションであるかを明確にするために一般的にフルワードが採用されている.-\/-no-の形にすることで否定形のオプションを作成することも可能である.ロングオプションはショートオプションのように複数のオプションを1つにまとめることは不可能であり,1つ1つをスペースで区切る必要がある.

ショートオプションを設定する際,英字1文字しか使用することができないので基本的には対応づけられたロングオプション(そのオプションがどのような動作を行うのかを表す単語)の頭文字であることが多く,慣れている人であればショートオプションを使うことで素早くアプリケーションを動かすことが可能である.しかし,用いるアプリが1つであるとは限らないし,全てのアプリのショートオプションを統一するのも困難である.またinitialとinstallなど,頭文字が重複してしまう2つのオプションがある場合,-iというショートオプションがinitialを意味するのかinstallを意味するのかを判断するのは,初心者には容易ではなく,混乱を引き起こしかねない.そのため,ヒューマンエラーを引き起こしてしまったり,学習コストがかかってしまったりすることがある.

ショートオプションの場合,複数のオプションを一つにまとめることができると記述したが,これは引数を必要としないオプションの場合である.引数を必要とするオプションの場合,2文字目以降の英字は引数扱いになってしまう.例えば上に示した-latにおいて,-lが引数を必要とするオプションであれば,-latはatという引数が与えられた-lという風に解釈されてしまうので注意が必要である.そういった点においてもショートオプションは初心者にとって扱いにくい形式であると言える.

    \section{基本的事項}\label{ux57faux672cux7684ux4e8bux9805}

\subsection{my\_help}\label{my_help}

\subsubsection{目的}\label{ux76eeux7684}

my
helpとは,ユーザー独自のマニュアルを作成することができるユーザメモソフトである.これは,terminalだけを用いて簡単に起動,編集,削除などをすることができるため,非常に便利である.さらに,そのマニュアルは自分ですぐに編集,参照することができるので,メモとしての機能も果たし
ている.これにより,プログラミング初心者が,頻繁に使うコマンドやキーバインドなどをいちいちweb
browserを立ち上げて調べるのではなく,terminal上で即座に取得できるため,プログラム開発を集中することが期待される.

また,正式なマニュアルは英語で書かれていることが多く,初心者には理解するのが困難である.my\_helpを使用すれば,自分なりのマニュアルを作成することができるので,仕様を噛み砕いて理解することができる.tarminal上でいつでもメモを参照できるため,どこにメモをしたかを忘れるリスクも軽減される

しかし,my
helpにある問題のうちの一つとして挙げられるのがサブコマンドの記述法である.

\subsubsection{コマンド}\label{ux30b3ux30deux30f3ux30c9}

clean exeディレクトリの中のファイルを全て削除する.

delete NAME NAMEのメモを削除する.

edit NAME NAMEのメモを編集する.

help COMAND COMANDのヘルプを表示する.

init NAME my helpの雛形をNAMEにコピーする.

install\_local
gemがそのPC内にあるとき,そのgemをコマンドとして実行できるようにする.rake
buildでパッケージ化されたヘルプを人に渡して,渡されたものをその人がinstall
localすることでパッケージ化されているヘルプを展開する.

list メモのリストを表示する

make 新しいメモを作成する.

version my\_helpのバージョンを表示する.

\subsection{optparse}\label{optparse}

ここにはoptparseの説明を書く.

今回の研究対象のmy
helpは,optparseで実装されている.optparseはRubyでコマンドラインのオプションを操作するためのライブラリである.optparseが操作するオプションは,下記のonメソッドで設定する.

\begin{Shaded}
\begin{Highlighting}[]
\KeywordTok{def}\NormalTok{ execute}
  \OtherTok{@argv}\NormalTok{ << ’--help’ }\KeywordTok{if} \OtherTok{@argv}\NormalTok{.size==}\DecValTok{0}
\NormalTok{  command_parser = }\DataTypeTok{OptionParser}\NormalTok{.new }\KeywordTok{do}\NormalTok{ |opt|}
\NormalTok{    opt.on(’-v’, ’--version’,’show␣program␣}\DataTypeTok{Version}\NormalTok{.’ ) \{ |v|}
\NormalTok{      opt.version = }\DataTypeTok{MyHelp}\NormalTok{::}\DataTypeTok{VERSION}
\NormalTok{      puts opt.ver}
\NormalTok{    \}}
\NormalTok{    opt.on(’-l’, ’--list’, ’list␣specific␣helps’)\{list_helps\}}
    \CommentTok{#中略}
  \KeywordTok{end}
  \CommentTok{#中略}
\KeywordTok{end}
    
\KeywordTok{def}\NormalTok{ list_helps}
\CommentTok{#中略}
\KeywordTok{end}
\CommentTok{#後略}
\end{Highlighting}
\end{Shaded}

第1引数はショートオプションで,-aや-dのような形で設定する.同様にして,第2引数はロングオプションを表し,-\/-addや-\/-deleteのように,第3引数はそのオプションの説明文で,helpで表示される説明文を設定する.後ろのブロックには,そのオプションが指定された場合に実行されるコードを記述する
{[}2{]}.しかしこのライブラリでは自然言語に近い,ハイフンなしのサブコマンドを実装するには相当な書き換えが必要となる.

メソッドの引数でオプションを定義し,引数が指定された時の処理をブロックで記述する.ブロックの引数にはオプションが指定されたことを示すtrueが渡される.onメソッドが呼ばれた時点ではオプションは実行されず,定義されるだけである.parseが呼ばれた際,コマンドラインにオプションが登録されていれば実行される.

オプション定義の際,スペースの後に任意の文字を追加すると,そのオプションは引数を受け取るオプションになる.その文字に{[}{]}をつけることで引数は必須でなくなる.また引数がハイフンで始まる場合,オプションとの間にハイフンを2つ挟むことで引数として認識される.

helpとversionのサブコマンドはデフォルトで作成される.

参考になりそうなところを見てoptparseについての説明.
長所短所.thorの長所短所と比較して図.あらかじめ描いてもいい.

\subsection{thor}\label{thor}

上と同じ.

本研究ではoptparseの代わりのライブラリとしてThorの採用を検討する.Thorは,コマンドラインツールの作成を支援するライブラリであり,gitやbundlerのようにサブコマンドを含むコマンドラインツールを簡単に作成することができる
{[}3{]}.Thorには以下のような特徴がある. 1.
コマンドラインオプションのパーズやサブコマンドごとのヘルプを作るなどの面倒な作業を簡単にこなすことができ,手早くビルドツールや実行可能なコマンドを作成できる
{[}4{]}. 1. 特殊なDSL(Domain Specific
Language)を使わずにメソッドを定義することで処理を記述するため,テストを行いやすい
{[}4{]}. 1.
optparseでは作成することが困難な,マイナスを伴わない(自然言語に近い)サブコマンドを実装することが可能である.

\begin{Shaded}
\begin{Highlighting}[]
\NormalTok{desc }\StringTok{'list, --list'}\NormalTok{, }\StringTok{'list specific helps'}
\NormalTok{    map }\StringTok{"--list"}\NormalTok{ => }\StringTok{"list"}
    \KeywordTok{def}\NormalTok{ list}
\NormalTok{      print }\StringTok{"Specific help file:\textbackslash{}n"}
\NormalTok{      local_help_entries.each\{|file|}
\NormalTok{        file_path=}\DataTypeTok{File}\NormalTok{.join(}\OtherTok{@local_help_dir}\NormalTok{,file)}
\NormalTok{        help = }\DataTypeTok{YAML}\NormalTok{.load(}\DataTypeTok{File}\NormalTok{.read(file_path))}
\NormalTok{        print }\StringTok{"  }\OtherTok{#\{}\NormalTok{file}\OtherTok{\}}\StringTok{\textbackslash{}t:}\OtherTok{#\{}\NormalTok{help[}\StringTok{:head}\NormalTok{][}\DecValTok{0}\NormalTok{]}\OtherTok{\}}\StringTok{\textbackslash{}n"}
\NormalTok{      \}}
    \KeywordTok{end}
\end{Highlighting}
\end{Shaded}

optparseではonメソッドでコマンドの登録を行い,その後のdefでコマンドの振る舞いを定義している.それに対してThorは登録と定義を同時に行うことができるのでコードが書きやすい.また,Thorを継承したクラスのパブリックメソッドがそのままコマンドになるので非常に簡単にコマンドを作成することが可能である.また,Thorはコマンドを作成した時点で,自動的にヘルプを生成してくれルため,楽である.コマンドの指定なしで実行するとヘルプを表示する.

長所短所と説明.optparseと同じく.
なんでthorなのか,動機,根拠.サブコマンドとか. 論理的に言えるように.

Thorではクラスを作ったらメソッドがそのままサブコマンドの処理になるので便利.optparseではそれができないので不便.

\subsection{awesome}\label{awesome}

ここにawesomeの和訳を入れても良い.

    \section{研究内容}\label{ux7814ux7a76ux5185ux5bb9}

\subsection{thor化に伴う書き換え(具体的な作業)}\label{thorux5316ux306bux4f34ux3046ux66f8ux304dux63dbux3048ux5177ux4f53ux7684ux306aux4f5cux696d}

ここは汎用的にどう書き換えたらthor化できるかまとめてると良さそう.
苦労した事とか書くといい.

どうゆう書き換えを行ったのか. thor化しようと思ったら
どうゆう流れでコマンドが呼び出されているか.(optとthor)

optparseからthorへの書き換えの際,第一に必要となるのは,\textasciitilde{}.gemspecファイル内にThorがインストールされるように変更を加えることである.

\begin{verbatim}
spec.add_development_dependency "thor"
\end{verbatim}

この記述は,作成しているコマンドラインツールがThorに依存性を持つことを意味している.optparseで書いた場合は,Thorに依存性を持つ必要がなく,当然このような記述はされていないので,上の一文を書き加える必要がある.また,bundle
updateをターミナル上で実行することで,ローカルでのコマンドラインツールを実行した際にThorが適用されるようにアップデートされる.そうすることで,Thorの記述が実際にソースファイルで使用できるようになる.

\subsection{thorによるサブコマンド}\label{thorux306bux3088ux308bux30b5ux30d6ux30b3ux30deux30f3ux30c9}

次に,コマンドが呼ばれる流れについて説明する.

\begin{enumerate}
\def\labelenumi{\arabic{enumi}.}
\tightlist
\item
  コマンドを実行する.
\item
  コマンドを実行すると,exeディレクトリの中にあるコマンド名と同じ名前のファイル(以降コマンドファイルと呼ぶ)が実行される.
\item
  コマンドファイル内でlibディレクトリ内のソースファイルをrequireしておき,クラス内のコマンドを解析する関数を呼び出す.
\item
  ソースファイル内に書かれた処理が実行される.
\end{enumerate}

このような流れでコマンドが呼び出され,処理が行われる.Thorとoptpaseの差異は4.におけるコマンドの解析による処理関数への中継の方法である.

\subsection{exeディレクトリの書き換え}\label{exeux30c7ux30a3ux30ecux30afux30c8ux30eaux306eux66f8ux304dux63dbux3048}

exeディレクトリの中にあるコマンドファイルについて書き換えを行う.まずはoptparseを使用した際のコマンドファイルの記述を記載する.

\begin{Shaded}
\begin{Highlighting}[]
\KeywordTok{#!/usr/bin/env ruby                                                                                                               }
\NormalTok{require }\StringTok{"my_help_opt"}

\DataTypeTok{MyHelp}\NormalTok{::}\DataTypeTok{Command}\NormalTok{.run(}\DataTypeTok{ARGV}\NormalTok{)}
\end{Highlighting}
\end{Shaded}

optparseの場合,MyHelp::Command.run(ARGV)というクラス関数を実行することでコマンドを呼び出している.しかし,Thorはoptparseのようにrun(ARGV)を用いず,start(ARGV)というクラス関数を実行してコマンドを呼び出す.よって,runをstartに書き換える作業が必要になる.以下にThorを使用した際のコマンドファイルの記述を記載する.

\begin{Shaded}
\begin{Highlighting}[]
\KeywordTok{#!/usr/bin/env ruby                                                                                                               }
\NormalTok{require }\StringTok{"my_help_thor"}

\DataTypeTok{MyHelp}\NormalTok{::}\DataTypeTok{Command}\NormalTok{.start(}\DataTypeTok{ARGV}\NormalTok{)}
\end{Highlighting}
\end{Shaded}

\subsection{libディレクトリ}\label{libux30c7ux30a3ux30ecux30afux30c8ux30ea}

まず,self.runについてであるが,Thorでは使用されないのでこの一文は削除する.次にinitializeであるが,こちらについては変更が必要である.そもそもinitializeとはmy\_helpを動かすのに必要なディレクトリがあるかどうかを調べるメソッドであり,なければここで作ることができる.optparseのinitializeではargv={[}{]}となっているところをThorでは(アスタリスク)argvとする必要がある.また,大きな違いはsuperの有無である.optparseではsuperは必要ないのだが,Thorでは必要になる.これは,

\subsection{その他オプション}\label{ux305dux306eux4ed6ux30aaux30d7ux30b7ux30e7ux30f3}

optparseではコマンド実行の際,引数の解析を行い,その引数に合わせた関数を呼び出す,という手順で動作している.その手順を記述した関数がexecuteである.

\begin{Shaded}
\begin{Highlighting}[]
\KeywordTok{def}\NormalTok{ execute}
      \OtherTok{@argv}\NormalTok{ << }\StringTok{'--help'} \KeywordTok{if} \OtherTok{@argv}\NormalTok{.size==}\DecValTok{0}
\NormalTok{      command_parser = }\DataTypeTok{OptionParser}\NormalTok{.new }\KeywordTok{do}\NormalTok{ |opt|}
\NormalTok{        opt.on(}\StringTok{'-v'}\NormalTok{, }\StringTok{'--version'}\NormalTok{,}\StringTok{'show program Version.'}\NormalTok{) \{ |v|}
\NormalTok{          opt.version = }\DataTypeTok{MyHelp}\NormalTok{::}\DataTypeTok{VERSION}
\NormalTok{          puts opt.ver}
\NormalTok{        \}}
        
\NormalTok{        opt.on(}\StringTok{'-l'}\NormalTok{, }\StringTok{'--list'}\NormalTok{, }\StringTok{'list specific helps'}\NormalTok{)\{list_helps\}}
\NormalTok{        opt.on(}\StringTok{'-e NAME'}\NormalTok{, }\StringTok{'--edit NAME'}\NormalTok{, }\StringTok{'edit NAME help(eg test_help)'}\NormalTok{)\{|file| edit_help(file)\}}
\NormalTok{        opt.on(}\StringTok{'-i NAME'}\NormalTok{, }\StringTok{'--init NAME'}\NormalTok{, }\StringTok{'initialize NAME help(eg test_help).'}\NormalTok{)\{|file| init_help(file)\}}
\NormalTok{        opt.on(}\StringTok{'-m'}\NormalTok{, }\StringTok{'--make'}\NormalTok{, }\StringTok{'make executables for all helps.'}\NormalTok{)\{make_help\}}
\NormalTok{        opt.on(}\StringTok{'-c'}\NormalTok{, }\StringTok{'--clean'}\NormalTok{, }\StringTok{'clean up exe dir.'}\NormalTok{)\{clean_exe\}}
\NormalTok{        opt.on(}\StringTok{'--install_local'}\NormalTok{,}\StringTok{'install local after edit helps'}\NormalTok{)\{install_local\}}
\NormalTok{        opt.on(}\StringTok{'--delete NAME'}\NormalTok{,}\StringTok{'delete NAME help'}\NormalTok{)\{|file| delete_help(file)\}}
      \KeywordTok{end}
      
      \KeywordTok{begin}
\NormalTok{        command_parser.parse!(}\OtherTok{@argv}\NormalTok{)}
      \KeywordTok{rescue}\NormalTok{=> eval}
\NormalTok{        p eval}
      \KeywordTok{end}
\NormalTok{      exit}
\KeywordTok{end}
\end{Highlighting}
\end{Shaded}

しかしThorの場合,executeのような間を取り持つ関数を用意する必要がなく,関数自体をコマンドとして登録していく形をとっているので,この関数は不必要である.以下にThorでの関数宣言を記載する.optparseno場合,-iというコマンドで呼び出されている処理をinitというコマンドで呼び出されるように書き換えたのが以下の関数である.

\begin{Shaded}
\begin{Highlighting}[]
\NormalTok{desc }\StringTok{'init NAME, --init NAME'}\NormalTok{, }\StringTok{'initialize NAME help(eg test_help).'}
\CommentTok{# 関数についての説明,ここがヘルプで表示される.}
\NormalTok{map }\StringTok{"--init"}\NormalTok{ => }\StringTok{"init"}
\CommentTok{# --オプションでも呼び出せるようにしてある.}
\KeywordTok{def}\NormalTok{ init(file)}\CommentTok{# 上にずらずらと書いているが,実際にオプション名として参照しているのは関数名らしい.}
\CommentTok{#以下は変更なし}
\NormalTok{    p target_help=}\DataTypeTok{File}\NormalTok{.join(}\OtherTok{@local_help_dir}\NormalTok{,file)}
    \KeywordTok{if} \DataTypeTok{File}\NormalTok{::exists?(target_help)}
\NormalTok{        puts }\StringTok{"File exists. rm it first to initialize it."}
\NormalTok{        exit}
    \KeywordTok{end}
\NormalTok{    p template = }\DataTypeTok{File}\NormalTok{.join(}\OtherTok{@default_help_dir}\NormalTok{,}\StringTok{'template_help'}\NormalTok{)}
    \DataTypeTok{FileUtils}\NormalTok{::}\DataTypeTok{Verbose}\NormalTok{.cp(template,target_help)}
\KeywordTok{end}
\end{Highlighting}
\end{Shaded}

このように書き換えることで,Thorを使用したオプションの設定を行うことができるのだが,実際に動かしてみるとエラーが表示されることがある.ここで表示されるエラーは,コマンドとして設定されていない関数があることについての警告文である.その関数はコマンドとして使用しないということを明確に記述することでこの警告を消すことが可能である.

\begin{Shaded}
\begin{Highlighting}[]
\NormalTok{no_commands }\KeywordTok{do}

    \CommentTok{#この間にコマンド設定しない関数を記述}

\KeywordTok{end}
\end{Highlighting}
\end{Shaded}

\subsection{thorで実装できない部分と対処}\label{thorux3067ux5b9fux88c5ux3067ux304dux306aux3044ux90e8ux5206ux3068ux5bfeux51e6}

optparseとthorの両立に成功したので,それに触れる.ここ真剣に.

    \section{比較}\label{ux6bd4ux8f03}

ここでoptparse版とthor版の比較をする.

    \section{総括}\label{ux7dcfux62ec}


    % Add a bibliography block to the postdoc
    
    
    
    \end{document}
