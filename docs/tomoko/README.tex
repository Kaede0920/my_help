% Created 2018-09-13 木 13:23

  \documentclass[a4j,twocolumn]{jsarticle}
  \usepackage[dvipdfmx]{graphicx}
  \usepackage[utf8]{inputenc}
  \usepackage[T1]{fontenc}
  
\usepackage[utf8]{inputenc}
\usepackage[T1]{fontenc}
\usepackage{fixltx2e}
\usepackage{graphicx}
\usepackage{longtable}
\usepackage{float}
\usepackage{wrapfig}
\usepackage{rotating}
\usepackage[normalem]{ulem}
\usepackage{amsmath}
\usepackage{textcomp}
\usepackage{marvosym}
\usepackage{wasysym}
\usepackage{amssymb}
\usepackage[dvipdfmx]{hyperref}
\usepackage{pxjahyper}
\tolerance=1000
\setlength{\textheight}{275mm}
\headheight 5mm
\topmargin -30mm
\textwidth 185mm
\oddsidemargin -15mm
\evensidemargin -15mm
\pagestyle{empty}



\title{memoソフトmy\_helpの開発}
\author{情報科学科 \hspace{5mm} 27014520 \hspace{5mm} 山田智子}
\date{}

\hypersetup{
  pdfkeywords={},
  pdfsubject={},
  pdfcreator={Emacs 25.3.1 (Org mode 8.2.10)}}
\begin{document}

\maketitle



\section{はじめに}
\label{sec-1}
\subsection{背景}
\label{sec-1-1}
我々はアクティブラーニングを実現するためのシステム構築に必要なmy\_helpというmemoソフトを開発する.



\subsection{アクティブラーニング}
\label{sec-1-2}
テキストや教授者から知識を得るのではなく,自らも参加者になって知識を得る.

学習には2つの方法がある.
AM(acquisition metaphor)とPM(participation metaphor)と呼ばれるものである.
下記に具体的な内容を示す.

\begin{center}
\begin{tabular}{lll}
 & acquisition metaphor & participation metaphor\\
\hline
学習目標 & 個々を豊かにする & 共同体の構築\\
\hline
学習とは? & 何かを獲得する(acquisition)) & 参加者(participant)となる\\
\hline
学習者(student) & 受容者(消費者),再構築者  & 周辺にいる参加者,徒弟\\
\hline
教授者(teacher) & 供給者,まとめ役,媒介者 & 実践や論考の修得者\\
 &  & \\
\hline
知識,概念 & 資産,所有物,一般商品 & 実践,論考,活動の一側面\\
 & (個人のあるいは公共の) & \\
\hline
知るとは & 持つ,所有すること & 所属する,参加する,\\
 &  & コミュニケートすること\\
\hline
\end{tabular}
\end{center}



\section{手法}
\label{sec-2}
\subsection{my\_helpの開発コンセプト}
\label{sec-2-1}
my\_help自体は,TODOファイルやmemoである.org-modeを利用しているので,長文やlatex,htmlにも対応している.
my\_helpにemacsのキーバインド(振る舞い)を書くことによって振る舞い管理システム( BMS,Behavior Management System)となる.

ノートにmemoをするが,無くしてしまうorどこにやったか分からなくなる.->my\_helpを使うとそれがなくなる.

人間の脳が知識を何かの拍子に思い出す,キーワードが手がかりとなって違う分類,別の階層の記憶,知識を思い出す様に,システム上でも同じことができるシステム.人間の脳において知識が分類され,階層に分けられている(と仮定して)様に,コンピュータの中でも分類化,階層化された知識を直交補空間的に知識を呼び出すことを可能にするtool.

my\_helpに共有するシステムをのせると強制することができる.


\subsection{my\_helpの特徴}
\label{sec-2-2}
\begin{itemize}
\item emacsのMarkdownであるorg-modeを利用.
\item org-modeで作成した文章はemacs以外でも利用できる.
\item 例えば,githubでは.mdと同じ様に.orgに対応している.
\item org-modeのexport機能を利用すればHTMLやLaTexなど様々なフォーマットに変換可能.
\end{itemize}
(\url{https://qiita.com/dwarfJP/items/594a8d4b0ac6d248d1e4})
\begin{itemize}
\item my\_helpを使うにあたり,emacsとorg-modeの使い方をmasterしなければならない.
\item CUI/CLIのようにterminal上で動かす.
\item commandで呼び出すとすぐに起動する.
\end{itemize}

\subsection{my\_helpの振る舞い}
\label{sec-2-3}
terminal上でmy\_help file名と打つと起動する.
\begin{description}
\item[{delete}] delete HELP\_NAME help
\item[{edit}] edit HELP\_NAME help, emacsを使ってorg-modeで編集
\item[{list}] list all helps, specific HELP, or item
\item[{new}] make new HELP\_NAME help
\end{description}

TODOとして使うとき,DONEのやつはarchive(書庫)に入れる.いらなくなったから.

\subsection{my\_help 課題,開発目標}
\label{sec-2-4}
my\_helpを共有することで,知識の効率的な共有を進める.

\begin{itemize}
\item my\_helpは時々動かない.
\item どういうerror?
\item versionをあげた.
\end{itemize}
\begin{verbatim}
<       p target_help = File.join(@local_help_dir,file+'.org')
<       if local_help_entries.member?(file+'.org')
<         system "emacs #{target_help}"
<       else
<         puts "file #{target_help} does not exits in #{@local_help_dir}."
<         puts "init #{file} first."
---
>       target_help = File.join(@local_help_dir,file)
>       ['.yml','.org'].each do |ext|
>         p target_help += ext if local_help_entries.member?(file+ext)
\end{verbatim}



\subsubsection*{内世界,外世界}
\label{sec-2-5-1}
\begin{itemize}
\item 内世界は自分の中の知識

\item 外世界はテキストや論文などの知識
\end{itemize}
% Emacs 25.3.1 (Org mode 8.2.10)
\end{document}
