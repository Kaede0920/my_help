\chapter{はじめに}\label{ux306fux3058ux3081ux306b}

\section{目的}\label{ux76eeux7684}

西谷研究室で使われているユーザメモソフト,my
helpの振る舞いを制御しているサブコマンドは,マイナスを付した省略記法が取られている.プログラミング初心者にとってこの省略記法は,覚えにくかったりわかりにくかったりするという問題があり,現在はフルワードを使った自然言語に近い記述法が多く用いられている.
そこで,本研究ではコマンドラインツール作成ライブラリを自然言語に近いサブコマンド体系を実装しやすいライブラリであるThorに変更する.rubyの標準ライブラリであるoptparseで作成されているmy
helpを,Thorによって書き直し,異なった2つライブラリで作成されたmy
helpの使用感を比較検討することを目的とする.

\section{コマンドの重要性}\label{ux30b3ux30deux30f3ux30c9ux306eux91cdux8981ux6027}

今日,複雑な機能を持つコマンドが増加している.そのようなコマンドは,サブコマンドを使用することで適切な動作を実行することが可能になる.例えばgitコマンドはサブコマンドなしでは意味をなさない.サブコマンドでどのような動作をするかが決まるので,サブコマンドを入力することで正常に動作するのである.

そもそも,サブコマンド(オプション)にはショートオプションとロングオプションの2種類がある.ショートオプションはハイフンの後に英字1字を付けた形式のもので,-aや-vなどといったものがショートオプションである.また,ショートオプションは2つ以上のオプションを1つにまとめて実行することもできる.例えば,-l,-a,-tの3つのオプションを1つにまとめて-latとして実行することが可能である.それに対してロングオプションは,ハイフン2つの後に英字2字以上を付けることができる形式である.例えば-\/-allや-\/-versionなどである.ロングオプションは英字を2字以上使用することができるので,どのようなオプションであるかを明確にするために一般的にフルワードが採用されている.-\/-no-の形にすることで否定形のオプションを作成することも可能である.ロングオプションはショートオプションのように複数のオプションを1つにまとめることは不可能であり,1つ1つをスペースで区切る必要がある.

ショートオプションを設定する際,英字1文字しか使用することができないので基本的には対応づけられたロングオプション(そのオプションがどのような動作を行うのかを表す単語)の頭文字であることが多く,慣れている人であればショートオプションを使うことで素早くアプリケーションを動かすことが可能である.しかし,用いるアプリが1つであるとは限らないし,全てのアプリのショートオプションを統一するのも困難である.またinitialとinstallなど,頭文字が重複してしまう2つのオプションがある場合,-iというショートオプションがinitialを意味するのかinstallを意味するのかを判断するのは,初心者には容易ではなく,混乱を引き起こしかねない.そのため,ヒューマンエラーを引き起こしてしまったり,学習コストがかかってしまったりすることがある.

ショートオプションの場合,複数のオプションを一つにまとめることができると記述したが,これは引数を必要としないオプションの場合である.引数を必要とするオプションの場合,2文字目以降の英字は引数扱いになってしまう.例えば上に示した-latにおいて,-lが引数を必要とするオプションであれば,-latはatという引数が与えられた-lという風に解釈されてしまうので注意が必要である.そういった点においてもショートオプションは初心者にとって扱いにくい形式であると言える.

    