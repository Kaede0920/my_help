\chapter{総括}\label{ux7dcfux62ec}

今回の研究では,ユーザメモソフトであるmy
helpのthorによる書き換えを行った.この書き換えによる成果は以下の通りである.

\begin{enumerate}
\def\labelenumi{\arabic{enumi}.}
\item
  オプションの記法を-(ハイフン)を用いた省略記法からフルワードを用いた自然言語に近い記法に変更した.これにより,初心者でも直感的にオプションを使用することができ,学習コストの削減,学習効率の向上に繋がると考えられる.
\item
  thorを用いることによってoptparseを用いた時よりもソースコードを短くすることに成功した.
\end{enumerate}

    