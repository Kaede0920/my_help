\chapter{基本的事項}\label{ux57faux672cux7684ux4e8bux9805}

\section{my\_help}\label{my_help}

\subsection{my\_helpの目的}\label{my_helpux306eux76eeux7684}

my\_helpとは,ユーザー独自のマニュアルを作成することができるユーザメモソフトである.これは,terminalだけを用いて簡単に起動,編集,削除などをすることができるため,非常に便利である.さらに,そのマニュアルは自分ですぐに編集,参照することができるので,メモとしての機能も果たし
ている.これにより,プログラミング初心者が,頻繁に使うコマンドやキーバインドなどをいちいちweb
browserを立ち上げて調べるのではなく,terminal上で即座に取得できるため,プログラム開発を集中することが期待される.

メモやtodoリストの作成が行えることや,保存場所を共通化することでどこでも立ち上げることができることなど,emacsのorg-modeと類似している点がいくつか存在する.しかし,明確な相違点も存在する.org-modeはemacsを起動させなければならないが,my\_helpはemacsを起動させる必要がなくterminalで編集することが可能である.また,org-modeを使用するとなるとorg-mode独自のコマンドを学ぶ必要があり,学習コストがかかってしまう.my\_helpにはその必要がなく,非常に単純な操作でアプリを使用することができるので,org-modeの使い方を理解していない初心者にとって使いやすいものとなっている.

また,アプリやプログラミング言語などの正式なマニュアルは英語で書かれていることが多く,初心者には理解するのが困難である.my\_helpを使用すれば,自分なりのマニュアルを作成することができるので,仕様を噛み砕いて理解することが可能である.tarminal上でいつでもメモを参照できるため,どこにメモをしたかを忘れるリスクも軽減される.

\subsection{使用法}\label{ux4f7fux7528ux6cd5}

インストールする方法だが,gemの標準とは少し方法が異なっている.
まず,githubにあるmy\_helpのリポジトリをフォーク,クローンすることでローカル(ネットワークに繋がれていない環境)でもmy\_helpを操作することができるようになる.

\begin{quote}
git clone git@github.com:daddygongon/my\_help.git
\end{quote}
これ以降の作業はbundleにて行っていく.

\begin{quote}
bundle update
\end{quote}
を実行することでmy\_help.gemspecに記述されている必要なgemsがbundleされる.ここでCould
not locate
Gemfileとエラーが出た場合は、Gemfileのある場所を探し、その配下に移動してから再びコマンドを入力する.

\begin{quote}
bundle exec exe/my\_help
\end{quote}
でmy\_helpに用意されているコマンドを参照することができる.デフォルトでemacs\_helpというemacsのヘルプが用意されている.これはemacs\_helpの他に,省略形のe\_hでも表示されるようになっている.

次に,独自のヘルプを作成する方法であるが,まず,

\begin{quote}
bundle exec exe/my\_help -i new\_help
\end{quote}
とすることでnew\_helpという名前のヘルプが作成され,そこにテンプレートが格納される.また,

\begin{quote}
bundle exec exe/my\_help -e new\_help
\end{quote}
で,自分の好きなように編集することができる.ヘルプが完成したら,

\begin{quote}
bundle exec exe/my\_help -m
\end{quote}
とすることでexeディレクトリにnew helpが追加され,new
help,n\_hが使用可能になるという手順である.

    \section{option}\label{option}

今日,複雑な機能を持つコマンドが増加している.そのようなコマンドは,オプション(サブコマンド)を使用することで適切な動作を実行することが可能になる.例えばgitコマンドはオプションなしでは意味をなさない.オプションでどのような動作をするかが決まるので,オプションを入力することで正常に動作するのである.

\subsection{optionの種類}\label{}

optionの種類については以下のように述べられている.

\begin{quote}
そもそも,オプションにはショートオプションとロングオプションの2種類がある.ショートオプションはハイフンの後に英字1字を付けた形式のもので,-aや-vなどといったものがショートオプションである.また,ショートオプションは2つ以上のオプションを1つにまとめて実行することもできる.例えば,-l,-a,-tの3つのオプションを1つにまとめて-latとして実行することが可能である.それに対してロングオプションは,ハイフン2つの後に英字2字以上を付けることができる形式である.例えば-\/-allや-\/-versionなどである.ロングオプションは英字を2字以上使用することができるので,どのようなオプションであるかを明確にするために一般的にフルワードが採用されている.-\/-no-の形にすることで否定形のオプションを作成することも可能である.ロングオプションはショートオプションのように複数のオプションを1つにまとめることは不可能であり,1つ1つをスペースで区切る必要がある\cite{awesome}pp.14-5.
\end{quote}

ショートオプションを設定する際,英字1文字しか使用することができないので基本的には対応づけられたロングオプション(そのオプションがどのような動作を行うのかを表す単語)の頭文字であることが多く,慣れている人であればショートオプションを使うことで素早くアプリケーションを動かすことが可能である.しかし,用いるアプリが1つであるとは限らないし,全てのアプリのショートオプションを統一するのも困難である.またinitialとinstallなど,頭文字が重複してしまう2つのオプションがある場合,-iというショートオプションがinitialを意味するのかinstallを意味するのかを判断するのは,初心者には容易ではなく,混乱を引き起こしかねない.そのため,ヒューマンエラーを引き起こしてしまったり,学習コストがかかってしまったりすることがある.

ショートオプションの場合,複数のオプションを一つにまとめることができると記述したが,これは引数を必要としないオプションの場合である.引数を必要とするオプションの場合,2文字目以降の英字は引数扱いになってしまう.例えば上に示した-latにおいて,-lが引数を必要とするオプションであれば,-latはatという引数が与えられた-lという風に解釈されてしまうので注意が必要である.そういった点においてもショートオプションは初心者にとって扱いにくい形式であると言える.

Command Line Applicationのオプションの記述方法には幾つもの流儀があるようで
何らかの標準があるわけではない.
しかし,それら全てに対応することはできず,なんらかの
基準に従ってオプション記法を解釈する必要がある.

ここでは, "Build awesome command-line application in ruby 2"に従って
オプション記法と用語をまとめておく.

コマンドラインの基本形は,

\begin{quote}
ls -lat dir\_name
\end{quote}
というように

\begin{quote}
executabel options arguments
\end{quote}
での形であった.これは,GNU標準に基本構造が記載されている.

その後,幾つかのswitchやflagを組み合わせて,複雑な
命令を解釈できるようにするに従って,

\begin{quote}
grep -\/-igonre-case -C 4 "some string" /tmp
\end{quote}
などとほぼ呪文のような形態となってきた.

その後,command line suiteと呼ばれる一群のcommand line
applicationが登場した.典型的なのはgitである\cite{awesome}pp.14-5.

gitはlinuxのバックアップを分散処理するために, Linus
Torvaldsが開発したものであるが,いくつもの
機能に従ってそれぞれ個別のコマンドが用意されていた.
それぞれ,git-commitとかgit-fetchなどであった.
それがある時,すべてをまとめてsuiteとして
パッケージし直され,1つのまとまったcommandとして提供された.
すなわち,git commitやgit fetchなどである\cite{awesome}p.16.

\subsection{Command Line Interface}\label{}

Graphical User Interface(GUI)はコンピュータグラフィックスとポインティングデバイスを用いて操作を行う方法である.
それに対しCommand Line Interface(CLI)はCharacter User Interface(CUI)とも呼ばれ,キーボードからの入力と文字による情報の表示だけを用いて操作を行う方法である.

 "Build awesome command-line application in ruby 2"の中で,

\begin{quote}
グラフィカルユーザーインターフェース(GUI)はいろんな点で素晴らしいものです. 冷たくまたたくカーソルの殺伐とした輝きよりも, GUIはとりわけ初心者にとても優しいものです. でも,それには犠牲が伴います.GUIの熟練者になるには,奥義のようなキーボードショートカットを学ぶ必要があります. そうだとしても,あなたは生産性と効率の限界にぶち当たります. GUIはスクリプトして自動化しにくいことで悪名高いし,それができたとしても,あなたのスクリプトは移植しにくい傾向にあります\cite{awesome}pp.3-4.
\end{quote}

と述べられており,GUIの限界について示唆している.また,

\begin{quote}
あなたのアプリをインストールした後,それを用いたユーザーが最初に体験するのは実際のコマンドラインインターフェイスでしょう.もしそのインターフェイスが難しく,直感的でなく,もしくは,とても醜いならば,たくさんの自信を呼び起こすことはできないし,ユーザーはそれを使って明確で簡潔な目的を達成するのに苦労するでしょう.逆に,使いやすければ,あなたのインターフェイスはアプリケーションに観衆と鋭さを与えてくれるでしょう\cite{awesome}pp.3-4.
\end{quote}

とも述べられている.
