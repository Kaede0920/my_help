
\subsection{my\_helpについて}
my\_helpは本研究室の西谷が開発したものです.
my\_helpとはユーザメモソフトであり,CUIスキルの習得を助けてくれます.
tarminal上で簡単に提示させることができるため,プログラミングに集中することができるといった特徴があります.
また,自分の見やすいように初心者でも簡単に編集することができ,すぐに参照できるメモとしても使うことができます.

\subsubsection{my\_helpの開発動機と特徴}
以下はmy\_helpのREADMEからの抜粋です[2].

\paragraph{概要}
my\_help - CUI(CLI)ヘルプのUsage出力を真似て,user独自のhelpを作成・提供するgem.

\paragraph{開発動機}
CUIやshell, 何かのプログラミング言語などを習得しようとする初心者は,
commandや文法を覚えるのに苦労します.少しのkey(とっかかり)があると
思い出すんですが,うろ覚えでは間違えて路頭に迷います.問題点は,

\begin{itemize}
\item manは基本的に英語
\item manualでは重たい
\item いつもおなじことをwebで検索して
\item 同じとこ見ている
\item memoしても,どこへ置いたか忘れる
\end{itemize}
などです.

\paragraph{特徴}
これらをgem環境として提供しようというのが,このgemの目的です.
仕様としては,

\begin{itemize}
\item userが自分にあったmanを作成
\item 雛形を提供
\begin{itemize}
\item おなじformat, looks, 操作, 階層構造
\end{itemize}
\item すぐに手が届く
\item それらを追加・修正・削除できる
\end{itemize}
hikiでやろうとしていることの半分くらいはこのあたりのことなの
かもしれません.memoソフトでは,検索が必要となりますが,my\_helpは
key(記憶のとっかかり)を提供することが目的です.

\subsubsection{my\_helpのインストール}
githubに行ってdaddygongonのmy\_helpをforkします.

\begin{enumerate}
\item git clone git@github.com:daddygongon/my\_help.git
\item cd my\_help
\item rake to\_yml
\item rake clean\_exe
\item [sudo] bundle exec exe/my\_help -m
\item source ~/.zshrc or source ~/.cshrc
\item my\_help -l
\item rake add\_yml
\end{enumerate}
\subsubsection{my\_helpの更新}
git hubを用いてmy\_helpを新しくします.

\begin{enumerate}
\item git remote -vをする(remoteの確認).
\item (upstreamがなければ)git remote add upstream git@github.com:gitname/my\_help.git
\item git add -A
\item git commit -m 'hogehoge'
\item git push upstream master(ここで自分のmy\_helpをupstreamに送っとく)
\item git pull origin master(新しいmy\_helpを取ってくる)
\end{enumerate}
次にとってきた.ymlを~/.my\_helpにcpする.

\begin{enumerate}
\item cd my\_helpでmy\_helpに移動.
\item cp hogehoge.yml ~/.my\_help
\end{enumerate}
それを動かすために
(sudo)bundle exec ruby exe/my\_help -mをする.
ここで過去にsudoをした人はpermissionがrootになっているので,sudoをつけないとerrorが出ます.
(sudoで実行していたら権限がrootに移行される)

\begin{enumerate}
\item 新しいターミナルを開いて動くかチェックする.
\end{enumerate}
