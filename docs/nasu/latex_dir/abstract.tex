プログラム開発では,統合開発環境がいくつも用意されているが,多くの現場では,terminal上での開発が一般的です.
ところが,プログラミング初心者はterminal上でのcharacter user interface(CUI)を苦手としています.
この不可欠なCUIスキルの習得を助けるソフトとして,ユーザメモソフトmy\_helpがruby gemsに置かれています.
このcommand line interface(CLI)で動作するmy\_helpは,helpをterminal上で簡単に提示するものです.
また,初心者が自ら編集することによって,すぐに参照できるメモとしての機能を提供しています.
これにより,プログラム開発に集中できることが期待でき,初心者のスキル習得が加速することが期待できます.
しかし,Ruby gemsとして提供されているこのソフトは,動作はするがテストが用意されていません.
今後ソフトを進化させるために共同開発を進めていくには,仕様や動作の標準となるテスト記述が不可欠となります.
そこで,本研究では,ユーザメモソフトであるmy\_helpのテストを開発することを目的とします.
本研究では、テスト駆動開発の中でも,ソフトの振る舞いを記述します
Behavior Driven Development(BDD)に基づいてテストを記述していきます.
Rubyにおいて、BDD環境を提供する標準的なフレームワークであるCucumberとRSpecを用いて,
my\_helpがどのような振る舞いをするのかを記述します.
Cucumberは自然言語で振る舞いを記述することができるため,ユーザにとって,わかりやすく振る舞いを確認することができます.
ここでは,実際にBDDの流れにそって,my\_helpのある一例を取り上げてのテスト開発を進めていきます.
また,my\_helpの頻繁に使うコマンドの振る舞いを記述しています.
このことにより,来年の本研究室の研究生のスキルがより向上することが期待されると考えています.

